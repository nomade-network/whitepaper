\documentclass{sistedes}
\usepackage[utf8]{inputenc}
\usepackage[english]{babel}
\usepackage{biblatex}

\addbibresource{refs.bib}

\begin{document}

\title{Nomade: A decentralised Social Media Platform}

\author{Clemente Sutil \inst{1}}

\institute{Founder of Nomade \\ \url{www.nomade.social} \\ \email{clementeserranosutil@gmail.com}}

\maketitle

\small

\keywords{
  Web3, social media, Decentralised Social Media (DeSo), blockchain.
}

\begin{abstract}
Social media business model is making our society sick and polarised. Heralded in its infancy for giving a voice to the voiceless, for connecting the world as a whole, and as a multiplier for democracy and truth is now derided by many for facilitating the spread of disinformation, acting as a tool for radicalisation, hijacking user attention, and for using a business model built on invasion of user privacy. There is thus an urgent need to re-think the way we benefit from the power of social media. With Nomads we propose a decentralised, censorship-resistant platform which aims to democratise the way people interact and share content with others on the Internet.
\end{abstract}

\section{Introduction}
\subsection{The promise of social media}
There are manifest and clear benefits to using social media that accrue to individuals and society at large. For individual users, social media can serve as a link to both local and international communities, improve their sense of connectedness, help them to build social capital and profit from the resources of their own social networks [1,4,16]. It can provide access and communication tools for cultural movements, advocacy groups, political parties, and even governments. See, for example, social media effects on Arab spring [17] or the umbrella movement in Hong Kong [18]. See Singer \& Brooking [19] for a detailed discussion on the political power of social media and its impact on society. Additionally, social media can also be of help to disaster management and emergency services [20].

At the moment of writing, more than four billion people use a social media service [21]. The sheer number of users and the countless hours being spent on these platforms [21,22] underline the notion that people find value in using social media.

In summary, it is clear that social media helps to build social capital [35] and offers a number of benefits to individuals [1,4], organisations [36,37] and social movements [2,3] that are worth preserving.

\subsection{The actual ad-driven business model}
Social media has grown much too fast, and the initial techno-optimism has recently started giving way to critical assessments of its wide-ranging drawbacks. Much of this critique can be traced to 'monetisation of surveillance capacities' \cite{monetizationOfSurveillandeCapacitiesModel} as the prevalent business model underlying most social media companies. This model entails that users exchange their personal data for usage allowance of a platform. The data thus collected is used to predict manifold psycho-social characteristics on a fine granular level, which in turn is used for advertisements that gain in value by employing microtargeting and personalised persuasion [38,39]. For empirical insights into the effectiveness of psychological targeting see exemplary works by Matz et al. [40] and Zarouali et al. [39]. Monetisation of the deep insights into the lives of social media users gained from their digital footprints has made tech-giants such as Meta (formerly known as Facebook) among the richest companies in the world. The platforms’ vested interest in maximising both the collection of a user's personal data and the display of advertisements has led over many years of AB-testing to highly immersive online-platforms designed to foster long online stays and high engagement on the part of the users [41].

The dark side of the data business model behind social media services has become apparent over time. For instance, design elements such as personalised news-feeds might not only lead to addictive use of social media services, but might also create political filter bubbles among some of the users [14], thus putatively narrowing the world view especially of those users who get news exclusively via social media [42]. This business model also pays little regard to widely treasured notions of privacy and abuses it as a matter of course [38,43]. Additionally, it has been observed that algorithms driving engagement can enrage people [11,44] and it is well known that fake news represents a tremendous problem on social media platforms [45,46]. Recent political ramifications such as the storming of the US-capitol in the aftermath of the last US presidential elections [47], where social media is known to have played an important role [48], have put the spread of fake news under even greater spotlight. Lastly, as social media is controlled by private corporations, independent researchers have only very limited or no access to data and algorithms that they require to study other dangerous tendencies in social media usage [49], [50], [51]

The only way forward seems to be to expand discussion on what a healthier and better version of social media could be.

\subsection{Information is controlled by the ones who have more social capital}
Social media platforms profess that their ultimate goal is to connect everyone together in a social network (with the aim to harvest a maximum of digital footprints) but there are some perils that may be intrinsic to connecting large groups of humans [52]. The graph that results from the “friendship connections” in social networks can be described as a “small world network” [53]. This leads to the effect that everybody is connected to everyone else with just a few steps and, importantly, some nodes (users) in the network come to have far more edges (connections) than the average user. But this also means that the whole network structure is dependent on these hyper-connected users with disproportionate number of edges, who are often also hyperactive, and are therefore central to the flow of communication [11]. Having such power over information flow is one reason why so many people want to become influencers and why hyper-active users are causing so much problems in social networks [11]. For instance, the spread of fake news has been attributed to only a few highly connected and hyper-active users [54].

In contrast to the early vision of web 2.0 that was aimed at allowing everybody to speak out, today only few have the power to be heard on social media.

\subsection{Communication is an echo-chamber}
Social media communication is an echo-chamber, we need more diverse discussions!
If we see social media from the perspective of flow of information, there are clear and well-studied negative effects that result from the way communication happens on social media [55,56]. Users can end up in interconnected echo chambers where they reinforce each other’s pre-existing attitudes by seeing and discussing similar, attitude-aligning content over and over again [56]. Different opinions are presented but mostly filtered through partisan comments of other accounts from the same echo-chamber. Additionally, what goes “viral” in the end is the most simplistic version of any public opinion carefully tailored with emotional triggers [57]. Surely, sometimes virality caused by the recommendation systems can be desirable, e.g., when the best cat meme reaches everyone who is interested in cats. But such viral content is also competing with, and presumably reducing the outreach of all the other relevant content because it is put into one’s ‘personalised feed’ or ‘timeline’. In political settings, it has been shown that these dynamics can lead to polarisation [56]. Social media communication, as curated by the major platforms, has thus become very similar to the rainbow press, transferring everything in simple and often scandalous messages.

\subsection{Massively “over-fitting” of the signal of users’ preferences}
As discussed previously, many of the problems caused by social media platforms can be traced to their business model. The incentive to find out which advertisements fit to which user has meant that platforms try to maximise interaction at all costs. Social media platforms are using machine learning algorithms to predict which content will likely be “most relevant” for the user. But relevance here falls down to a simple metric: interaction. Social platforms count all the likes, shares, retweets etc. and try to match the content to the users in a way that the frequency of interaction is maximised – an approach coined “meaningful interaction”. From the companies’ perspective, this approach maximises revenue – more interactions reveal more preferences of the user and also lead to more time to show advertising. From the user perspective, “meaningful interaction” seems to present content that might capture attention at first glance but is essentially shallow – fast reactions are encouraged over thoughtful replies; content that triggers the user seems to be highlighted. The final result of this “meaningful interaction” is, unnecessary and often negative, excitement, arousal, and commotion [55].

It is also important to note that the automated recommendation systems used by the social platforms are producing these results without any human intervention. Especially deep learning has been very efficient in finding complex patterns in content and users’ meta-data that can be harvested without even understanding what the pattern is. Algorithms from the field of computer vision can label pictures and videos and find similarities. Big models like GPT-3 and BERT from the field of natural language processing are capable of revealing very subtle patterns in written or spoken content [58]. The better these algorithms work to trigger the users, the more obvious it becomes that these systems seem destined to follow a vicious cycle – they are trained on the interactions of the users that were already manipulated by the recommendation system [11]. This reinforcement loop makes these algorithms very efficient in maximising user interaction in the interest of the platform’s business model. Using a term from the machine learning community, social platforms are massively “over-fitting” the signal of users’ preferences. Instead of focusing on the so-called “meaningful interaction”, algorithms could instead be trained to present content that triggers “healthy behaviour” [14] like long-reads, lasting friendships, and nontoxic debates.

\section{Nomads}
While the currently dominant platforms can and should reform themselves, it is doubtful whether a healthier social media could ever be built while keeping the structures and incentives under which their parent companies operate. A more effective alternative could be to emancipate the idea of social media from the currently prevailing companies, take a new path and look for alternative models of running and structuring social media networks and platforms.

Nomads is a peer-to-peer governed, open-source, ad-free and censorship-resistant community owned online social media platform.

Its decentralised nature does away with dependency on any single, centralised entity. Secondly, they are based on open-source protocols and software which ensures transparency and gives a far wider range of developers the opportunity to innovate and try newer models and ideas. Importantly, the underlying code being open-source also means that algorithms that decide what appears in a user’s feed are not secret and proprietary but open for anyone to examine. Thirdly, they tend to be crowdfunded and ad-free and hence are able to provide better control and ownership of personal data to the user and of creative content to the creator. Fourthly, they tend to be community owned and put the responsibility of operation and content moderation on the community of users.

By removing centralised control, Nomads empowers users to connect and share content freely and without interference from corporate interests or algorithmic manipulation.

\section{Limited outreach}
Nomads aims to change the existing power structures that are monopolised by a few hyperactive users and distribute the control over information flow across many users to make the network more dynamic. 

This can be achieved by limiting the outreach (the number of edges/connections) a single user can have.

\section{Content viralisation redesign}
As a possible solution, instead of recommendation systems pushing viral content, social media communication could be explicitly tailored to prioritise content for diverse interests differentiated by topics, regions and consciously chosen preferences. This could be done by including diversity as a necessity in recommendation algorithms and giving users better and meaningful control over what appears in their feed.

\subsection{A new business model}
Social media platforms remain free for the users, but as the internet truism goes, “if it’s free, you are the product”. Much as the monetization of surveillance is not desirable [13], money and resources required needed for running social media platforms have to come from somewhere. Alternative payment models are hence an important part of any discussion to restart social media. WhatsApp, in its initial years, offered a fair and exemplary monetary model. It was free for the first year, with a quite minimal \$0.99 fee charged once every subsequent year. The company neither showed advertisements, nor sold data to private interests. The fee was fair and affordable for users and sustainable for the company. But soon after the company was acquired by Meta in 2014, this model was predictably discontinued. At the same time, there are downsides to alternatives that require direct payment by users. It remains unclear, now that users are used to social media platforms being ‘free’, how many would be willing to pay directly with their own money [72,73] and how many would rather keep paying with their personal data. It does not seem desirable to split the public into those who can afford digital privacy and those who can’t.

One possible solution might be to expand the definition of public service media to include social media [74,75]. If social media can rightfully be seen as a public utility essential to democratic functioning, it stands well within the stated objectives and mandates of public service broadcasting laws in most liberal democracies [74]. Public broadcasters are funded in a variety of ways, which allow them to work solely for public service, from an obligatory television license fee, individual contributions, government funding and limited advertising. As discussed earlier, many of the problems associated with social media arise from the business model of maximizing engagement and selling surveillance to advertisers. A public digital social network would not have many of the problems of private social media companies simply by the virtue of not having their business model – no advertisers to maximize engagement for and sell data to, and no investors pushing for growth at any cost.

\section{Conclusions}
Social media platforms were initially praised for connecting people worldwide, amplifying unheard voices, and enhancing democracy. However, today's social media business model is often criticised for spreading disinformation, promoting radicalisation, invading privacy, and polarising society. We need to rethink the way social media benefits us.

Nomads proposes a solution to these issues. It's designed to democratise online interactions and content sharing. By removing centralised control, Nomads empowers users to connect and share content freely and without interference from corporate interests or algorithmic manipulation. This approach promotes healthy discourse, encourages meaningful connections, and enables us to connect with one another on a truthly horizontal way.

Now more than ever, we need to embrace the decentralisation of social media to safeguard the future of our society. With Nomads, we can move towards a healthier, more democratic future where everyone has a voice, and meaningful conversations can take place without the interference of external forces.

\printbibliography

\end{document}